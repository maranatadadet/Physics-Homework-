\documentclass[12pt]{article}
\usepackage{epsfig}
\usepackage{times}
\usepackage{amsmath}
\usepackage{multirow}
\usepackage[shortlabels]{enumitem}
\renewcommand{\topfraction}{1.0}
\renewcommand{\bottomfraction}{1.0}
\renewcommand{\textfraction}{0.0}
\setlength {\textwidth}{6.6in}
\hoffset=-1.0in
\oddsidemargin=1.00in
\marginparsep=0.0in
\marginparwidth=0.0in                                                                              
\setlength {\textheight}{9.0in}
\voffset=-1.00in
\topmargin=1.0in
\headheight=0.0in
\headsep=0.00in
\footskip=0.50in                                        
\setcounter{page}{1}
\begin{document}
\def\pos{\medskip\quad}
\def\subpos{\smallskip \qquad}
\newfont{\nice}{cmr12 scaled 1250}
\newfont{\name}{cmr12 scaled 1080}
\newfont{\swell}{cmbx12 scaled 800}
%%%%%%%%%%%%%%%%%%%%%%%%%%%%%%%%%%%%%%%%%%%%%%%%%%%%%%%%%%%%
\thispagestyle{empty}
\begin{enumerate}
\begin{center}
 \textbf{\large PHYS 20323/60323: FALL 2019 - LaTeX Example}
\end{center}
  \item Consider a particle confined in a two-dimensional infinite square well
  \begin{equation*}
    V(x,y) =
    \begin{cases}
      0, &  0 \leq x \leq a,  0 \leq y \leq a \\
      \infty, & \text{otherwise}
    \end{cases}
  \end{equation*}
  The eigenfuctions have the form:
  \begin{equation*}
  \Psi(x,y) = \frac{2}{a}sin(\frac{n\pi x}{a})sin(\frac{m\pi y}{a})
  \end{equation*}
  with the corresponding energies being given by:
  \begin{center}
  \begin{equation*}
      E_{nm} = (n^{2} + m^{2}) + \frac{\pi^{2}\hbar}{2ma^{2}}
  \end{equation*}
   \end{center}
  \begin{enumerate}[(a)]
      \item (5 points) What are the levels of degeneracy of the five lowest energy levels?
      \item (5 points) Consider a perturbation given by:
      \begin{equation*}
          \hat{H}^{'} = a^{2}V_{0}\delta(x-\frac{a}{2})\delta(y-\frac{a}{2})
      \end{equation*}
      Calculate the first order correction to the ground state energy
  \end{enumerate}
  \item \textbf{The following questions refer to stars in the Table bellow.} \\
  Note: There may be multiple answers
  \begin{center}
 \begin{tabular}{ |l |l| l| l| l| l|}
 \hline
Name & Mass & Luminosity & Lifetime & Temperature & Radius\\ \hline
Zeta   & 60. M_{sun} &  10^{6} L_{sun} & 8.0 x 10^{5} years & &  \\ \hline
Epsilon   & 6.0 M_{sun}  &  10^{3} L_{sun} & & 20,000 K &    \\ \hline
Delta   & 2.0 M_{sun}   &  &  5.0 x 10^{8} years & & 2 R_{sun} \\ \hline
Beta & 1.3 M_{sun} & 3.5 L_{sun} & & & &\\ \hline
Alpha & 1.0 M_{sun} & & & & 1 R_{sun}\\ \hline
Gamma & 0.7 M_{sun} & & 4.5 x 10^{10} years & 5000 K & \\
\hline
\end{tabular}
\end{center}
 \begin{enumerate}[(a)]
     \item (4 points) Which of these stars will produce a planetary nebula at the end of their life.
     \item (4 points) Elements heavier than Carbon will be produces in which stars
 \end{enumerate}
\end{enumerate}
\end{document}